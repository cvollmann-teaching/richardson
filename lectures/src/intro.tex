% !TeX spellcheck = de_AT_frami
\section*{Überblick, Lernziele und Organisation}

Die Kerninhalte des praktisch orientierten zweiten Teils sind:
\begin{enumerate}
	\item Code:~\textbf{Programmieren mit Python in \pycharm}
	\item Text:~\textbf{Mathematische Textverarbeitung mit \LaTeX~in \texstudio}
	\item IT/Backup:~\textbf{Daten-Management} in der Rheinland-Pfalz-cloud \seafile\ und \textbf{Versionskontrolle} mit \git~auf dem \gitlab~Server der Universität Trier
\end{enumerate}
Wir lernen diese Komponenten nicht nur isoliert kennen, sondern werden uns sukzessive eine \textbf{integrierte Arbeitsumgebung} bauen, die das fließende Zusammenspiel dieser Komponenten erlaubt und Sie damit technisch vorbereitet auf zum Beispiel:
\begin{itemize}
	\item Numerik-Veranstaltungen (Code)
	\item Seminararbeiten (Vortrag, Text, Code, IT/Backup)
	\item Praktikumsbericht (Text, IT/Backup)
	\item Bachelor- und Master-Arbeiten (Text, Code, IT/Backup)
	\item Doktorarbeit und Forschung (Vortrag, Text, Code, IT/Backup,...)
\end{itemize}
Da wir beim Programmieren unumgänglich mit dem Computer arbeiten, werden wir auch ein Grundverständnis für IT-bezogene Themen aufbauen und dabei Begrifflichkeiten klären wie:
\begin{itemize}
	\item Betriebssystem (Linux, Windows, macOS)
	\item GUI vs CLI; Benutzeroberfläche samt Dateimanager VS. Shell/Terminalemulator
	\item Virtualisierung (VM, Container, VPN)
	\item Netzwerke, VPN, ssh, nfs, rsync, webdav, www/internet und https, cloud
\end{itemize}
Projekt:
\begin{itemize}
	\item Um das Erlernte an einem Fallbeispiel praktisch anzuwenden, werden wir uns ein bestimmtes mathematisches Problem vornehmen und dieses mit geeigneten numerischen Methoden lösen.
	\item Genauer werden wir einen iterativen Löser für lineare Gleichungssysteme implementieren. Das wird bereits ein Vorgeschmack auf die spannenden Numerik-Veranstaltungen im weiteren Studiumsverlauf.
	\item Anschließend wenden wir unseren Löser auf ein Netzwerk an, um dessen PageRank zu berechnen.
\end{itemize}
\clearpage
\textbf{Der Ablauf}
\begin{itemize}
	\item \textbf{Mathematisches Problem}
	\item \textbf{Numerische Methode} entwickeln/auswählen
	\item \textbf{Projektaufbau} planen
	\begin{itemize}
		\item Modularisierung Code und Text
		\item Ordnerstruktur
	\end{itemize}
	\item Hinweis zur Organisation und Prüfung
	\item \textbf{IT/Backup}: 
	\begin{itemize}
		\item  Hardware, Betriebssystem, Netzwerke, Virtualisierung
		\item \seafile\ (\texttt{.seafile-data}): Ordnerstruktur in der Cloud anlegen 
	\end{itemize}
	\item \textbf{CODE}: Implementierung (in Python)
	\begin{itemize}
		\item \textbf{IDE} \pycharm~(\texttt{.idea})
		\item \textbf{Modularisierung} umsetzen:
		\begin{itemize}
			\item Formate für Vektoren und Matrizen: Python Listen und CSR Klasse
			\item ``Level 1 BLAS Routinen'': Skalarmultiplikation, Addition, Euklidische Norm
			\item ``Level 2 BLAS Rotuinen'': Matrix--Vektor Produkt
			\item ``LAPACK'': rel. Richardson aufbauend auf Level 1 und Level 2 Routinen
		\end{itemize}
		\item {\color{gray}\textbf{Versionskontrolle}: \git\ und \gitlab~ (\texttt{.git})}
		\item \textbf{Dokumentation}
		\item \textbf{Tests/Beispiele} für alle Komponenten
	\end{itemize}
	\item \textbf{TEXT}: Wissenschaftliche Arbeit verfassen
	\begin{itemize}
		\item \textbf{\LaTeX}, IWE \texstudio
		\item \textbf{Modularisierung} umsetzen	
		\item Numerische Ergebnisse auswerten: Tabelle und Grafik
	\end{itemize}
	\item {\color{gray}\textbf{Kontinuierliche Integration}: CODE und TEXT (Prozesse in Schleifen)
	\begin{itemize}
		\item Grafiken: \texttt{matplotlib} > \texttt{savefig} > \latexCommand{includegraphics}{imagefile}
		\item Tabellen: Python Liste > Markdown > \texttt{pandoc} > \latexCommand{input}{table.tex} 
	\end{itemize}
}
\end{itemize}

