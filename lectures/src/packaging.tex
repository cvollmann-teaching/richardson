\section{Software Packaging (skip)}

\textbf{Paketverwaltung (\textit{package manager}):}  Wenn wir Pakete \textit{nicht} aus \pycharm~heraus installieren, benutzen wir das Tool \ttt{\textbf{pip}}:
\begin{itemize}
	\item pip ist der Paket-Verwalter für Python. Mit pip können wir über Kommandozeilen Pakete aus dem \link{https://pypi.org/}{Python Package Index} (und mehr) (de-)installieren
	\item zB: $$\ttt{\$ pip3 install numpy}$$
	\item siehe auch das manual via $$\ttt{\$ man pip}$$
	\item Beachte: Abhängig vom Benutzer, der \texttt{pip} ausführt (ob als sudo oder nicht, aus der venv heraus oder nicht,...), landen die Pakete z.B. entweder in der systemweiten, der lokalen oder virtuellen Python-Umgebung (bei uns oben installiert \pycharm~die Pakte in unsere virtuelle Umgebung (venv))
\end{itemize}
