% !TeX spellcheck = de_AT_frami
\section{Secure Shell (SSH)}
Was ist SSH?
\begin{itemize}
	\item SSH ist ein Protokoll für \textit{verschlüsselte} Verbindungen zwischen zwei Rechnern.
	\item SSH kann die Kommandozeile eines entfernten Rechners lokal zur Verfügung stellen.
	\item SSH kann Dateien über das Netzwerk transferieren.
	\item SSH kann grafische Fenster des entfernten Rechners anzeigen.
\end{itemize}
Warum ist das für uns interessant?
\begin{itemize}
	\item Für rechenintensive Programme verwenden wir \textit{number cruncher} (Remote-Server mit viel Rechenleistung). Dafür müssen Dateien wie Quellcode, Inputdaten etc. verschlüsselt zwischen Rechnern transferiert werden.
	\item Wir verwalten unser Projekt auf \url{github.com}. Wir transferieren unsere Dateien sicher über einen verschlüsselten SSH--Tunnel zwischen diesem Server und unserer lokalen Maschine. 
\end{itemize}
Demo:
\begin{itemize}
	\item  ssh auf remote
\item remote: Programm ausführen
 \item scp zum kopieren von Daten
\end{itemize}
Weitere Bemerkungen:
\begin{itemize}
	\item SSH wurde erstmals 1995 von dem finnischen Informatiker Tatu Ylönen entwickelt und ist heute die gängigste Methode für den Zugriff auf Remote-Linux-Server.
	\item SSH ist das allgemeine Protokoll. Die de facto Standardimplementierung ist OpenSSH (1999 von den OpenBSD-Entwicklern als Open-Source-Software veröffentlicht).
	\item Client Server Modell
	\begin{itemize}
		\item client (lokal): ssh
		\item server (entfernt): sshd
	\end{itemize}
\end{itemize}
Wiederholung Public-Key-Verschlüsselungsverfahren (asymmetrisch)
\begin{itemize}
	\item Schlüsselpaar $(k_\text{pub}, k_\text{priv})$
	\begin{itemize}
		\item öffentlicher Schlüssel $k_\text{pub}$ kann geteilt werden
		\item privater Schlüssel $k_\text{priv}$ muss geheim gehalten werden
	\end{itemize}
		\item Verschlüsselung mit öffentlichem Schlüssel: $y = E(x, k_\text{pub})$
\item Entschlüsselung mit privatem Schlüssel: $D(y, k_\text{priv}) = x$
\item Vorteil gegenüber symmetrischer Verschlüsselung (bei der für $E$ und $D$ derselbe Schlüssel verwendet wird): Alice und Bob müssen sich vor Kommunikation nicht nachts im Wald treffen, um sich auf ein Geheimnis zu einigen
\item Berühmtes Beispiel: RSA (siehe etwa Zertifikate im Browser)
\end{itemize}
%
%
\subsection{Verwendung}
\textbf{Passwort-basierte Authentifizierung:}
\begin{verbatim}
$ ssh [USERNAME]@[IP_ADDRESS or HOSTNAME]
\end{verbatim}

\begin{itemize}
	\item IP\_ADDRESS or HOSTNAME: Postanschrift des entfernten Rechners
	\item USERNAME: Benutzername des Accounts auf dem entfernen Rechner
	\item Passwort-Prompt
\end{itemize}
%
~\\
\textbf{Schlüsselbasierte Authentifizierung:}
\begin{itemize}
	\item Erstellen eines Schlüsselpaars
	\begin{verbatim}
	ssh-keygen -t ed25519 -f [FILENAME] -C "your_email@example.com"
	\end{verbatim}
	-- Speicherort: \texttt{\$HOME/.ssh}\\
	-- Passphrase, um den privaten Schlüssel zu schützen
	\item Dateien anschauen
	\begin{verbatim}
	cd ~/.ssh
	ls -al
	cat [FILENAME]
	\end{verbatim}
	\item Schlüssel auf remote kopieren
	\begin{verbatim}
	ssh-copy-id -i [FILENAME] [IP_ADDRESS or HOSTNAME]	
	\end{verbatim}
	Server speichert den öffentlichen Schlüssel in \texttt{.ssh/authorized\_keys}
	\item Nun einfach:
	\begin{verbatim}
	$ ssh [USERNAME]@[IP_ADDRESS or HOSTNAME]
	\end{verbatim}
\end{itemize}
%
~\\ \textbf{ssh-agent (skip)}




\subsection{Quellen}
\begin{itemize}
	\item \url{https://www.digitalocean.com/community/tutorials/how-to-use-ssh-to-connect-to-a-remote-server}
	\item \url{https://en.wikipedia.org/wiki/Secure_Shell}
	\item \url{https://en.wikibooks.org/wiki/OpenSSH}
	\item \url{https://en.wikipedia.org/wiki/Public-key_cryptography#}
\end{itemize}