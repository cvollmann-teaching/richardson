% !TeX spellcheck = de_AT_frami
\newpage
\section{Code Dokumentation mit	\texttt{sphinx}}

Der folgende Abschnitt ist nur ein Leitfaden für die Präsentation im Unterricht. Einen geeigneten Einstieg finden wir in der offiziellen Dokumentation
\begin{center}
	\url{https://www.sphinx-doc.org/en/master/usage/quickstart.html}
\end{center}



%\hypertarget{verzeichnisstruktur-aufbauen}{%
%	\subsection{Verzeichnisstruktur
%		aufbauen}\label{verzeichnisstruktur-aufbauen}}
%
%\begin{itemize}
%	\item
%	kleines skript fürs verzeichnis struktur bauen?
%	\item
%	seafile library anlegen
%	\item
%	verzeichnisstruktur anlegen in dateimanager o.ä
%\end{itemize}
%
%\begin{verbatim}
%|-- code
%|-- docs
%|-- src
%|-- examples
%|-- output
%|-- main.py
%|-- README.rst
%\end{verbatim}
%
%\begin{itemize}
%	\item
%	verzeichnis code mit PyCharm öffnen
%	\item
%	python venv interpreter einrichten
%	\item
%	Docstring Format \textbf{str alt s \textbar{} Tools \textbar{} Python
%		Integrated Tools} \textbar{} Docstrings \textgreater{} Docstring
%	Format: NumPy
%	\item
%	in \texttt{src/}
%
%	\begin{itemize}
%		\item
%		\textbf{init}
%		\item
%		linalg.py
%
%		\begin{itemize}
%
%			\item
%			class csr\_matrix
%			\item
%			def axpy()
%			\item
%			def scal()
%			\item
%			def nrm()
%			\item
%			def csrmv()
%		\end{itemize}
%		\item
%		in \textbf{main.py}
%
%		\begin{itemize}
%
%			\item
%			import src
%			\item
%			if \textbf{name}\_
%			\item
%			\ldots{}.
%			\item
%			edit config: run with console
%		\end{itemize}
%		\item
%		linalg.py etwas ausbauen
%
%		\begin{itemize}
%
%			\item
%			\texttt{def\ tests()}\ldots{}
%		\end{itemize}
%		\item
%		in \textbf{main.py}
%
%		\begin{itemize}
%
%			\item
%			import src
%			\item
%			run tests
%		\end{itemize}
%		\item
%		\textbf{Terminal} Reiter
%
%		\begin{itemize}
%
%			\item
%			ggf. \texttt{source\ .venv/bin/active}
%
%			\begin{itemize}
%
%				\item
%				source runs code from file in current bash session
%			\end{itemize}
%			\item
%			sonst: nochmal in den Einstellungen schauen
%			\item
%			hier könnten wir mit pip pakete installieren
%
%			\begin{itemize}
%
%				\item
%				import numpy --\textgreater{} fails
%				\item
%				siehe .venv/lib/site-packages
%				\item
%				pip3 install numpy
%				\item
%				siehe .venv/lib/site-packages
%				\item
%				import numpy
%			\end{itemize}
%		\end{itemize}
%		\item
%		requirements.txt
%
%		\begin{itemize}
%
%			\item
%			ggf. plugin installieren
%			\item
%			dann import und requirements.txt updaten
%		\end{itemize}
%	\end{itemize}
%\end{itemize}


\begin{itemize}
	\item
	      Beliebige Python--Dokumentation im Webbrowser finden: Unten im Footer meist Info über verwendete Software
	\item
	      Allgemein wollen wir u.a.:
	      \begin{itemize}
		      \item
		            Automatisierung
		      \item
		            Information nur an einer Stelle und daher Synchronisation von Code und Dokumentation
		      \item ggf. mathematische Formeln (brauchen entsprechende Markup Language)
	      \end{itemize}
	\item Wir verwenden dazu in diesem Kurs die Software \texttt{sphinx} (geschrieben in Python)
	\item	Dokumentation: \url{https://www.sphinx-doc.org/en/master/index.html}
	\item	Installation
	      \begin{itemize}
		      \item
		            GUI: strg+alt+s: project, interpreter --\textgreater{} install
		            sphinx
		      \item
		            CLI: \texttt{\$ pip3 install sphinx}
	      \end{itemize}
\end{itemize}

\hypertarget{getting-started}{%
	\subsection{\texorpdfstring{Getting
			Started}{Getting Started}}\label{getting-started}}

\begin{itemize}
	\item
	      PyCharm Terminal (.venv) \$
	\item
	      Zum Beispiel Version checken: \texttt{sphinx-build\ -\/-version}
\end{itemize}
%
\texttt{in \texttt{code/}}
\begin{itemize}
	\item \texttt{\$ mkdir docs}
	\item \texttt{\$ sphinx-quickstart\ \textless{}PATH-to-ROOTDIR\textgreater{}}
	      (hier einfach \texttt{docs/})
	      \begin{itemize}
		      \item Separieren source and build [y]
		      \item name: ...
		      \item release: 0.1
	      \end{itemize}
	\item Nun mal	Verzeichnis \texttt{code/docs/} inspizieren
	      \begin{itemize}

		      \item
		            \texttt{code/docs/source/}
		            \begin{itemize}

			            \item
			                  hier liegen die Quelldateien um die Dokumentation zu steuern
			            \item
			                  \texttt{conf.py}

			                  \begin{itemize}

				                  \item
				                        hier können wir die Dokumentation konfigurieren
				                  \item
				                        u.a. wurden unsere Antworten der interaktiven Abfrage von \texttt{sphinx-quickstart} dort gespeichert
			                  \end{itemize}
			            \item
			                  \texttt{index.rst} ist unsere Startseite (unsere Hauptdatei/``Kleber'')
			                  \begin{itemize}

				                  \item
				                        rst wie reStructuredText (ähnlich wie markdown)
				                  \item
				                        Direktiven und deren Optionen
				                  \item
				                        Überschriften \texttt{===,\ -\/-\/-\/-}
				                  \item
				                        setze zB \texttt{..\ note::}
			                  \end{itemize}
		            \end{itemize}
		      \item
		            \texttt{code/docs/build/}
		            \begin{itemize}
			            \item Dieses Verzeichnis ist noch leer
			            \item Das ändern wir nun
		            \end{itemize}
		      \item
		            \textbf{build}

		            \begin{itemize}

			            \item
			                  \texttt{\$ sphinx-build\ -b\ html\ docs/source/\ docs/build/html}
			            \item
			                  in \texttt{build/} Verzeichnis schauen
			            \item
			                  Wir finden nun einen Ordner \texttt{html}
			            \item
			                  Wir öffnen die Datei \texttt{index.html} mit einer geeigneten
			                  Software (= Browser)
			            \item
			                  \texttt{index.rst} {[}in reStructuredText{]} wird zu \texttt{index.html}\\
			                  \textit{			The file index.rst created by sphinx-quickstart is the root
				                  document, whose main function is to serve as a welcome page and to
				                  contain the root of the ``table of contents tree'' (or toctree).
				                  alternativer build command ist nun: \texttt{make\ html} in docs/}
		            \end{itemize}
	      \end{itemize}

	\item
	      Bemerkung zu den \texttt{sphinx-}binaries:

	      \begin{itemize}

		      \item
		            Diese befinden sich in unserer virtuellen Python Umgebung unter:
		            venv\texttt{/code/.\textless{}VENV-NAME\textgreater{}/bin}
		      \item
		            siehe auch: \texttt{\$\ locate\ sphinx-*} (ggf.
		            \texttt{\$ sudo\ updatedb} notwendig) (alternativ:
		            \texttt{\$ type\ sphinx-*})
	      \end{itemize}
\end{itemize}

\subsection{Maßschneiderung in \texttt{conf.py}}

\textbf{\texttt{extensions\ =\ {[}{]}}}

\begin{itemize}

	\item
	      Erweiterungen stellen zusätzliche Funktionalitäten bereit.
	\item
	      Hier können wir u.a. leicht Schalter setzen, um das Erscheinungsbild zu
	      steuern.
\end{itemize}

\hypertarget{allgemeines-erscheinungsbild-html_theme}{%
	\subsection{\texorpdfstring{\textbf{Allgemeines Erscheinungsbild:
				\texttt{html\_theme}}}{Allgemeines Erscheinungsbild: html\_theme}}\label{allgemeines-erscheinungsbild-html_theme}}

\begin{itemize}
	\item
	      \textbf{builtin schemes}
	      \begin{itemize}
		      \item siehe:
		            \url{https://www.sphinx-doc.org/en/master/usage/theming.html\#builtin-themes}
		      \item
		            Beispiel: \texttt{classic}
		      \item
		            in \texttt{conf.py}:\\ \hspace*{0.5cm}
		            \texttt{html\_theme\ =\ \textquotesingle{}classic\textquotesingle{}}
		      \item Neu bauen:
		            \ldots{}\texttt{docs/\ \$\ make\ html}
	      \end{itemize}
	\item
	      \textbf{third-party schemes}
	      \begin{itemize}
		      \item siehe:
		            \url{https://sphinx-themes.org/}
		      \item
		            Beispiel: \texttt{sphinx-rtd-theme\ (read\ the\ docs)}
		      \item
		            Diese müssen extra installiert werden, da nicht in Standard-Sphinx-Installation
		            enthalten
		            \begin{itemize}

			            \item
			                  \texttt{pip\ install\ sphinx-rtd-theme} oder über GUI
		            \end{itemize}
		      \item
		            in \texttt{conf.py}:\\ \hspace*{0.5cm}
		            \texttt{html\_theme\ =\ \textquotesingle{}sphinx\_rtd\_theme\textquotesingle{}}
	      \end{itemize}
\end{itemize}


\subsection{Struktur ausbauen}

\begin{itemize}
	\item
	      Neue Unterseiten anlegen, zB \texttt{docs/source/usage.rst}
	      %	\begin{Shaded}
	      %		\begin{Highlighting}[]
	      %			\NormalTok{Überschrift}
	      %			\NormalTok{===========}
	      %
	      %			\NormalTok{Etwas Text, blabla}
	      %
	      %			\DataTypeTok{.. code-block:: console}
	      %
	      %			\DataTypeTok{   (.venv) $ pip install sphinx}
	      %			\end{Highlighting}
	      %			\end{Shaded}
	\item Neu bauen:
	      \ldots{}\texttt{docs/\ \$\ make\ html}
	\item
	      Warnung:\\
	      \texttt{checking\ consistency...\ /.../usage.rst:\ WARNING:\ document\ isn\textquotesingle{}t\ included\ in\ any\ toctree}\\
	      >> Diese Seite möchte noch verlinkt werden damit wir darauf zugreifen
	      können
	\item
	      In unserer Hauptdatei zum Inhaltsbaum hinzufügen:\\~\\
	      \texttt{.. toctree::}\\
	      \texttt{\hspace*{0.5cm}  usage}
\end{itemize}

\subsection{Synchronisation von Code und	Dokumentation:
	\texttt{autodoc}}
\begin{itemize}
	\item
	      Nun wollen wir in der Dokumentation den Code beschreiben und dabei auf
	      vorhandene docstrings zurückgreifen.
	\item Das ist eine zusätzliche Funktionalität, die durch eine Erweiterung
	      bereitgestellt wird.
	      \begin{itemize}
		      \item
		            genauer \texttt{autodoc}:
		            \url{https://www.sphinx-doc.org/en/master/usage/extensions/autodoc.html}
	      \end{itemize}
	\item
	      aktivieren in \texttt{docs/source/conf.py}
	      \begin{itemize}
		      \item
		            zur Liste hinzufügen:
		            \texttt{extensions\ =\ {[}\textquotesingle{}sphinx.ext.autodoc\textquotesingle{},\ {]}}
		      \item
		            sonst Fehler wie
		            \texttt{ERROR:\ Unknown\ directive\ type\ ``autofunction''.}
	      \end{itemize}
	\item
	      Nun können wir zum Beispiel folgende Direktiven verwenden:\\
	      \texttt{ .. autofunction:: }\texttt{src.linalg.axpy}\\
	      \texttt{ .. autoclass:: }\texttt{src.linalg.csr\_matrix}
\end{itemize}
\begin{itemize}
	\item
	      Wir bekommen weiterhin Fehler, da die relativen Pfade src.linalg nicht
	      erkannt werden. Daher den absoluten Pfad zu
	      \texttt{\textless{}PATH\textgreater{}/code/} dem .venv Interpreter
	      mitgeben. Anpassung in \texttt{conf.py}:	\\
	      \texttt{import}\texttt{ pathlib}\\
	      \texttt{import}\texttt{ sys}\\
	      \texttt{sys.path.insert(0, pathlib.Path(\_\_file\_\_).parents[2].resolve().as\_posix())}\\

	      >> Der String
	      \texttt{pathlib.Path(\_\_file\_\_).parents{[}2{]}.resolve().as\_posix()} sollte dann
	      genau den Pfad zu \texttt{/.../code/} enthalten
	\item
	      Wir bekommen weiterhin einen Fehler, da unsere docstrings nicht im
	      reStructuredText Format sind, sondern NumPy Format\\
	      >> Diesen Fehler beheben wir im nächsten Schritt
\end{itemize}


\subsection{docstring Style: reStructuredText, NumPy, Google}
\begin{itemize}
	\item Dokumentation lesen: \url{https://sphinxcontrib-napoleon.readthedocs.io/en/latest/index.html}
	\item
	      für NumPy Format: \texttt{sphinxcontrib-napoleon}
	\item
	      Installation: GUI oder in CLI via \texttt{(.venv)\ \$\ pip\ install\ sphinxcontrib-napoleon}
	\item
	      conf.py: extension hinzufügen
	\item Type Annotations: \url{https://sphinxcontrib-napoleon.readthedocs.io/en/latest/index.html#type-annotations}
	\item
	      Examples in docstring, Code-Zeile hinter
	      \texttt{\textgreater{}\textgreater{}\textgreater{}}
	\item Tipp:	Docstring Format in PyCharm Settings setzen:\\\textbf{str alt s \textbar{} Tools \textbar{} Python
		      Integrated Tools} \textbar{} Docstrings \textgreater{} Docstring
	      Format: NumPy
	\item Nun auch nochmal \textbf{Indices} schauen: global, module, Search	Page
\end{itemize}



\hypertarget{reference-guide-automatisch-alle-code-bestandteile-auflisten-mit-..-autosummary}{%
	\subsection{\texorpdfstring{Reference Guide: Automatisch alle
			Code-Bestandteile auflisten mit
			\texttt{..\ autosummary::}}{Reference Guide: Automatisch alle Code-Bestandteile auflisten mit .. autosummary::}}\label{reference-guide-automatisch-alle-code-bestandteile-auflisten-mit-..-autosummary}}

\begin{itemize}
	\item \texttt{'sphinx.ext.autosummary'}
	\item
	      API references/Reference Guide VS User Guide
	\item
	      Wir legen dazu eine neue Unterseite an, zum Beispiel: \texttt{api.rst}
	\item
	      toctree--Direktive in \texttt{index.rst} ergänzen
	\item
	      Die Direktive \texttt{..\ autosummary::} wird durch die Extension
	      \texttt{sphinx.ext.autosummary} freigeschaltet
	\item
	      In \texttt{docs/source/api.rst} zum Beispiel:\\
	      \begin{verbatim}
	Reference Guide
	===============

	.. autosummary::
	    :toctree: generated

	    src.linalg
	\end{verbatim}
\end{itemize}

\hypertarget{links-zwischen-quellcode-und-docs-source-docs}{%
\subsection{\texorpdfstring{Links zwischen Quellcode und docs:
\texttt{{[}source{]},\ {[}docs{]}}}{Links zwischen Quellcode und docs: {[}source{]}, {[}docs{]}}}\label{links-zwischen-quellcode-und-docs-source-docs}}

\begin{itemize}

	\item
	      extension aktivieren: \texttt{sphinx.ext.viewcode}
\end{itemize}

\hypertarget{crossreferences-links-innerhalb-der-dokumentation}{%
	\subsection{Cross--References: Links innerhalb der Dokumentation}\label{crossreferences-links-innerhalb-der-dokumentation}}

\begin{itemize}

	\item
	      Label setzen vor Überschrift \texttt{..\ \_LABEL-TAG:}
	\item
	      Referenz auf den markierten Abschnitt: 	:ref:\texttt{LINK-NAME\ \textless{}LABEL-TAG\textgreater{}}
\end{itemize}


%%%%
\subsection{Mathematische Formeln \texttt{..\ math::}}
Zum Beispiel können wir dann \LaTeX\ verwenden setzen:
\begin{verbatim}
.. math::

    x\in\mathbb{R}^n
\end{verbatim}


%%
\subsection{Andere Build-Formate}
Zum Beispiel können wir auch mit dem Builder \texttt{latex} eine PDF--Dokumentation erstellen:
\begin{itemize}
	\item
	      Im Verzeichnis \texttt{code/docs/}: \texttt{\$ make\ latexpdf}
	\item
	      Alternativ im Verzeichnis \texttt{code/}: \texttt{\$ sphinx-build\ -b\ latex\ docs/source/\ docs/build/latex/}
\end{itemize}
