\hypertarget{verzeichnisstruktur-aufbauen}{%
\section{Verzeichnisstruktur
aufbauen}\label{verzeichnisstruktur-aufbauen}}

\begin{itemize}
\item
  kleines skript fürs verzeichnis struktur bauen?
\item
  seafile library anlegen
\item
  verzeichnisstruktur anlegen in dateimanager o.ä
\end{itemize}

\begin{verbatim}
|-- code
    |-- docs
    |-- src
    |-- examples
    |-- output
    |-- main.py
    |-- README.rst
\end{verbatim}

\begin{itemize}
\item
  verzeichnis code mit PyCharm öffnen
\item
  python venv interpreter einrichten
\item
  Docstring Format \textbf{str alt s \textbar{} Tools \textbar{} Python
  Integrated Tools} \textbar{} Docstrings \textgreater{} Docstring
  Format: NumPy
\item
  in \texttt{src/}

  \begin{itemize}
  \item
    \textbf{init}
  \item
    linalg.py

    \begin{itemize}
    \tightlist
    \item
      class csr\_matrix
    \item
      def axpy()
    \item
      def scal()
    \item
      def nrm()
    \item
      def csrmv()
    \end{itemize}
  \item
    in \textbf{main.py}

    \begin{itemize}
    \tightlist
    \item
      import src
    \item
      if \textbf{name}\_
    \item
      \ldots{}.
    \item
      edit config: run with console
    \end{itemize}
  \item
    linalg.py etwas ausbauen

    \begin{itemize}
    \tightlist
    \item
      \texttt{def\ tests()}\ldots{}
    \end{itemize}
  \item
    in \textbf{main.py}

    \begin{itemize}
    \tightlist
    \item
      import src
    \item
      run tests
    \end{itemize}
  \item
    \textbf{Terminal} Reiter

    \begin{itemize}
    \tightlist
    \item
      ggf. \texttt{source\ .venv/bin/active}

      \begin{itemize}
      \tightlist
      \item
        source runs code from file in current bash session
      \end{itemize}
    \item
      sonst: nochmal in den Einstellungen schauen
    \item
      hier könnten wir mit pip pakete installieren

      \begin{itemize}
      \tightlist
      \item
        import numpy --\textgreater{} fails
      \item
        siehe .venv/lib/site-packages
      \item
        pip3 install numpy
      \item
        siehe .venv/lib/site-packages
      \item
        import numpy
      \end{itemize}
    \end{itemize}
  \item
    requirements.txt

    \begin{itemize}
    \tightlist
    \item
      ggf. plugin installieren
    \item
      dann import und requirements.txt updaten
    \end{itemize}
  \end{itemize}
\end{itemize}

\hypertarget{dokumentation-mit-sphinx}{%
\section{\texorpdfstring{Dokumentation mit
\texttt{sphinx}}{Dokumentation mit sphinx}}\label{dokumentation-mit-sphinx}}

\begin{itemize}
\item
  Dokumentation zeigen im Browser

  \begin{itemize}
  \tightlist
  \item
    unten im Footer meist Info über verwendete Software
  \end{itemize}
\item
  Allgemeine Herangehenweise

  \begin{itemize}
  \tightlist
  \item
    wir wollen u.a..:

    \begin{itemize}
    \tightlist
    \item
      Automatisierung
    \item
      Information nur an einer Stelle: Sync code und doc
    \item
      ggf. mathematische Formeln (brauchen entsprechende Markup
      Language)
    \end{itemize}
  \end{itemize}
\item
  Hier an einem Beispiel: sphinx (geschrieben in Python)
\item
  Dokumentation lesen:

  \begin{itemize}
  \tightlist
  \item
    https://www.sphinx-doc.org/en/master/index.html
  \item
    https://sphinxcontrib-napoleon.readthedocs.io/en/latest/index.html
  \end{itemize}
\item
  Installation

  \begin{itemize}
  \tightlist
  \item
    gui: strg+alt+s: project, interpreter --\textgreater{} install
    sphinx
  \item
    terminal: pip3 install sphinx
  \end{itemize}
\end{itemize}

\hypertarget{getting-started}{%
\subsection{\texorpdfstring{\textbf{Getting
Started}}{Getting Started}}\label{getting-started}}

\begin{itemize}
\item
  pycharm: terminal (.venv) \$
\item
  Intressehalber Version checken: \texttt{sphinx-build\ -\/-version}
\end{itemize}

\textbf{in \texttt{code/}}

\begin{itemize}
\item
  \texttt{sphinx-quickstart\ \textless{}PATH-to-ROOTDIR\textgreater{}}
  (hier einfach \texttt{docs/})

  \begin{itemize}
  \tightlist
  \item
    sep source and build: y
  \item
    name:
  \item
    release: 0.1
  \end{itemize}
\item
  Verzeichnis docs/ inspizieren

  \begin{itemize}
  \tightlist
  \item
    \texttt{source/}

    \begin{itemize}
    \tightlist
    \item
      hier liegen die Quelldateien um die Dokumentation zu steuern
    \item
      conf.py

      \begin{itemize}
      \tightlist
      \item
        hier können wir die Dokumentation konfigurieren
      \item
        u.a. wurden unsere Abfragen dort gespeichert
      \end{itemize}
    \item
      index.rst ist unsere Startseite (unsere Hauptdatei/``Kleber'')

      \begin{itemize}
      \tightlist
      \item
        rst wie reStructuredText (ähnlich wie markdown)
      \item
        Direktiven und deren Optionen
      \item
        Überschriften \texttt{===,\ -\/-\/-\/-}
      \item
        setze zB \texttt{..\ note::}
      \end{itemize}
    \end{itemize}
  \item
    \texttt{build/}

    \begin{itemize}
    \tightlist
    \item
      dieses Verzeichnis ist noch leer
    \item
      das ändern wir nun
    \end{itemize}
  \end{itemize}
\item
  \textbf{build}

  \begin{itemize}
  \tightlist
  \item
    \texttt{sphinx-build\ -b\ html\ docs/source/\ docs/build/html}
  \item
    in \texttt{build/} Verzeichnis schauen
  \item
    wir finden nun einen Ordner \texttt{html}
  \item
    wir öffnen die Datei \texttt{index.html}mit einer geeigneten
    Software (=Browser)

    \begin{itemize}
    \tightlist
    \item
      zB mal nach der Notiz suchen
    \end{itemize}
  \item
    \texttt{index.rst} {[}in reStructuredText{]} becomes index.html

    \begin{itemize}
    \tightlist
    \item
      The file index.rst created by sphinx-quickstart is the root
      document, whose main function is to serve as a welcome page and to
      contain the root of the ``table of contents tree'' (or toctree).
      alternativer build command ist nun: \texttt{make\ html} in docs/
    \end{itemize}
  \end{itemize}
\item
  Bemerkung zu den \texttt{sphinx-}binaries:

  \begin{itemize}
  \tightlist
  \item
    Diese befinden sich in unserer
    venv\texttt{/code/.\textless{}VENV-NAME\textgreater{}/bin}
  \item
    siehe auch: \texttt{\$\ locate\ sphinx-*} (ggf.
    \texttt{sudo\ updatedb} notwendig) (alternativ:
    \texttt{type\ sphinx-*})
  \end{itemize}
\end{itemize}

\hypertarget{mauxdfschneiderungcostumization-conf.py}{%
\subsection{\texorpdfstring{Maßschneiderung/Costumization
\texttt{conf.py}}{Maßschneiderung/Costumization conf.py}}\label{mauxdfschneiderungcostumization-conf.py}}

\textbf{\texttt{extensions\ =\ {[}{]}}}

\begin{itemize}
\tightlist
\item
  Erweiterungen stellen zusätzliche Funktionalitäten bereit
\item
  hier können wir leicht Schalter setzen, um das Erscheinungsbild zu
  steuern
\end{itemize}

\hypertarget{allgemeines-erscheinungsbild-html_theme}{%
\subsubsection{\texorpdfstring{\textbf{Allgemeines Erscheinungsbild:
\texttt{html\_theme}}}{Allgemeines Erscheinungsbild: html\_theme}}\label{allgemeines-erscheinungsbild-html_theme}}

\begin{itemize}
\tightlist
\item
  \textbf{builtin schemes}

  \begin{itemize}
  \tightlist
  \item
    https://www.sphinx-doc.org/en/master/usage/theming.html\#builtin-themes
  \item
    Zb \texttt{classic}
  \item
    in conf.py:
    \texttt{html\_theme\ =\ \textquotesingle{}classic\textquotesingle{}}
  \item
    \ldots{}\texttt{docs/\ \$\ make\ html}
  \end{itemize}
\item
  \textbf{third-party schemes}

  \begin{itemize}
  \tightlist
  \item
    https://sphinx-themes.org/
  \item
    zB \texttt{sphinx-rtd-theme\ (read\ the\ docs)}
  \item
    muss installiert werden (da nicht in standard sphinx installation
    enthalten)

    \begin{itemize}
    \tightlist
    \item
      \texttt{pip\ install\ sphinx-rtd-theme} oder über GUI
    \end{itemize}
  \item
    in conf.py:
    \texttt{html\_theme\ =\ \textquotesingle{}sphinx\_rtd\_theme\textquotesingle{}}
  \end{itemize}
\end{itemize}

\hypertarget{struktur-ausbauen}{%
\subsection{\texorpdfstring{\textbf{Struktur
ausbauen}}{Struktur ausbauen}}\label{struktur-ausbauen}}

\begin{itemize}
\item
  Neue Unterseiten anlegen, zB \texttt{docs/source/usage.rst}

  mit Inhalt:

\begin{Shaded}
\begin{Highlighting}[]
\NormalTok{Überschrift}
\NormalTok{===========}

\NormalTok{Etwas Text, blabla}

\DataTypeTok{.. code-block:: console}

\DataTypeTok{   (.venv) $ pip install sphinx}
\end{Highlighting}
\end{Shaded}
\item
  build
\item
  Warnung:
  \texttt{checking\ consistency...\ /.../usage.rst:\ WARNING:\ document\ isn\textquotesingle{}t\ included\ in\ any\ toctree}
\item
  Diese Seite möchte noch verlinkt werden damit wir darauf zugreifen
  können
\item
  in unserer Hauptdatei zum Inhaltsbaum hinzufügen:

\begin{Shaded}
\begin{Highlighting}[]
\DataTypeTok{.. toctree::}

\NormalTok{   usage}
\end{Highlighting}
\end{Shaded}
\end{itemize}

\hypertarget{synchronisation-von-code-und-dokumentation-autodoc}{%
\subsection{\texorpdfstring{\textbf{Synchronisation von Code und
Dokumentation:
\texttt{autodoc}}}{Synchronisation von Code und Dokumentation: autodoc}}\label{synchronisation-von-code-und-dokumentation-autodoc}}

\begin{itemize}
\tightlist
\item
  nun wollen wir in der Dokumentation den Code beschreiben und dabei auf
  vorhandene docstrings zurückgreifen
\item
  das ist eine zusätzliche Funktionalität, die durch eine Erweiterung
  bereitgestellt wird:

  \begin{itemize}
  \tightlist
  \item
    genauer \texttt{autodoc}:
    https://www.sphinx-doc.org/en/master/usage/extensions/autodoc.html
  \end{itemize}
\item
  aktivieren in docs/source/conf.py

  \begin{itemize}
  \tightlist
  \item
    zur Liste hinzufügen:
    \texttt{extensions\ =\ {[}\textquotesingle{}sphinx.ext.autodoc\textquotesingle{},\ {]}}
  \item
    sonst Fehler wie
    \texttt{ERROR:\ Unknown\ directive\ type\ "autofunction".}
  \end{itemize}
\item
  nun können wir zum Beispiel folgene Direktiven verwenden
\end{itemize}

\begin{Shaded}
\begin{Highlighting}[]
\DataTypeTok{ .. autofunction:: }\NormalTok{src.linalg.axpy}
 
\DataTypeTok{ .. autoclass:: }\NormalTok{src.linalg.csr_matrix}
\end{Highlighting}
\end{Shaded}

\begin{itemize}
\item
  Wir bekommen weiterhin Fehler, da die relativen Pfade src.linalg nicht
  erkannt werden, daher den absoluten Pfad zu
  \texttt{\textless{}PATH\textgreater{}/code/} dem .venv Interpreter
  mitgeben:

  \begin{itemize}
  \tightlist
  \item
    \textbf{Pfade zum code hinzufügen in conf.py}
  \end{itemize}

\begin{Shaded}
\begin{Highlighting}[]
\ImportTok{import}\NormalTok{ pathlib}
\ImportTok{import}\NormalTok{ sys}
\NormalTok{sys.path.insert(}\DecValTok{0}\NormalTok{, pathlib.Path(}\VariableTok{__file__}\NormalTok{).parents[}\DecValTok{2}\NormalTok{].resolve().as_posix())}
\end{Highlighting}
\end{Shaded}

  Der String
  \texttt{pathlib.Path(\_\_file\_\_).parents{[}2{]}.resolve().as\_posix()}sollte
  genau den Pfad zu \texttt{/.../code/} enthalten
\item
  Wir bekommen weiterhin einen Fehler, da docstring kein
  reStructuredText sind sondern NumPy Format
\end{itemize}

\hypertarget{docstring-style-restructuredtext-numpy-google}{%
\subsection{\texorpdfstring{\textbf{docstring Style:} reStructuredText,
NumPy,
Google}{docstring Style: reStructuredText, NumPy, Google}}\label{docstring-style-restructuredtext-numpy-google}}

\begin{itemize}
\tightlist
\item
  für NumPy Format: \texttt{sphinxcontrib-napoleon}
\item
  Installation: gui oder cli
  \texttt{shell\ \ (.venv)\ \$\ pip\ install\ sphinxcontrib-napoleon}
\item
  conf.py: extension hinzufügen
\item
  read doc of Napolean
\item
  TypeSetting
\item
  Examples in docstring, code via
  \texttt{\textgreater{}\textgreater{}\textgreater{}}
\end{itemize}

\textbf{Indices} Nun auch nochmal schauen: - global - module - Search
Page

\hypertarget{reference-guide-automatisch-alle-code-bestandteile-auflisten-mit-..-autosummary}{%
\subsection{\texorpdfstring{Reference Guide: Automatisch alle
Code-Bestandteile auflisten mit
\texttt{..\ autosummary::}}{Reference Guide: Automatisch alle Code-Bestandteile auflisten mit .. autosummary::}}\label{reference-guide-automatisch-alle-code-bestandteile-auflisten-mit-..-autosummary}}

\begin{itemize}
\item
  API references/Reference Guide VS User Guide
\item
  wir legen dazu eine neue Unterseite an:\texttt{api.rst}
\item
  toctree in \texttt{index.rst} ergänzen
\item
  Die Direktive \texttt{..\ autosummary::} wird durch die Extension
  \texttt{sphinx.ext.autosummary} freigeschaltet
\item
  in \texttt{docs/source/api.rst}

\begin{Shaded}
\begin{Highlighting}[]
\DataTypeTok{.. autosummary::}
    \FunctionTok{:toctree:}\NormalTok{ generated}

\NormalTok{    src.linalg}
\end{Highlighting}
\end{Shaded}
\end{itemize}

\hypertarget{links-zwischen-quellcode-und-docs-source-docs}{%
\subsection{\texorpdfstring{Links zwischen Quellcode und docs:
\texttt{{[}source{]},\ {[}docs{]}}}{Links zwischen Quellcode und docs: {[}source{]}, {[}docs{]}}}\label{links-zwischen-quellcode-und-docs-source-docs}}

\begin{itemize}
\tightlist
\item
  extension aktivieren: \texttt{sphinx.ext.viewcode}
\end{itemize}

\hypertarget{crossreferences-links-innerhalb-der-dokumentation}{%
\subsection{\texorpdfstring{\textbf{Cross--References: Links innerhalb
der
Dokumentation}}{Cross--References: Links innerhalb der Dokumentation}}\label{crossreferences-links-innerhalb-der-dokumentation}}

\begin{itemize}
\tightlist
\item
  label/tag setzen vor Überschrift \texttt{..\ \_LABEL-TAG:}
\item
  Referenz auf den ge-label-ten Abschnitt
  `:ref:\texttt{LINK-NAME\ \textless{}LABEL-TAG\textgreater{}}
\end{itemize}

\hypertarget{mathematische-formeln-..-math}{%
\subsection{\texorpdfstring{\textbf{Mathematische Formeln}
\texttt{..\ math::}}{Mathematische Formeln .. math::}}\label{mathematische-formeln-..-math}}

\hypertarget{andere-build-formate}{%
\subsection{\texorpdfstring{\textbf{Andere
Build-Formate}}{Andere Build-Formate}}\label{andere-build-formate}}

\begin{itemize}
\item
  in code/docs/ \texttt{make\ latexpdf}
\item
  oder in code/
  \texttt{sphinx-build\ -b\ latex\ docs/source/\ docs/build/latex/}
\end{itemize}
