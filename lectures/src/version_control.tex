\newpage~\\
\section{Versionskontrolle}
ssh,...
\begin{itemize}
	\item https://www.freecodecamp.org/news/learn-the-basics-of-git-in-under-10-minutes-da548267cc91/
	\item
\end{itemize}


vcs enable git\\
.config anpassen\\
add, commit, push
\subsection{Allgemein}
Was ist Versionskontrolle?\\


Bei der Versionskontrolle handelt es sich um ein System, das Änderungen an einer Datei oder einer Reihe von Dateien im Laufe der Zeit aufzeichnet, so dass Sie bestimmte Versionen später wieder aufrufen können. Im Idealfall können wir also jede Datei auf dem Computer der Versionskontrolle unterstellen.\\


Hier ist der Grund dafür:\\


Mit einem Versionskontrollsystem (VCS) können Sie Dateien in einen früheren Zustand zurückversetzen, das gesamte Projekt in einen früheren Zustand zurückversetzen, die im Laufe der Zeit vorgenommenen Änderungen überprüfen, sehen, wer etwas zuletzt geändert hat, das ein Problem verursachen könnte, wer ein Problem eingeführt hat und wann, und vieles mehr.\\


Es ist also sehr hilfreich für die Fehlersuche. Oder Sie können einfach sehen, welche Änderungen Sie im Laufe der Zeit an Ihrem Code vorgenommen haben.
\subsection{git}
Git ist ein Versionskontrollsystem zur Verfolgung von Änderungen an Computerdateien und zur Koordinierung der Arbeit an diesen Dateien durch mehrere Personen. Git ist ein verteiltes Versionskontrollsystem. \\


 jeder Benutzer "klont" eine Kopie eines Repositorys (eine Sammlung von Dateien) und hat den gesamten Verlauf des Projekts auf seiner eigenen Festplatte.\\



 Git hilft Ihnen auch, Code zwischen mehreren Personen zu synchronisieren. Stellen Sie sich also vor, Sie und Ihr Freund arbeiten gemeinsam an einem Projekt. Sie arbeiten beide an denselben Projektdateien. Nun nimmt Git die Änderungen, die Sie und Ihr Freund unabhängig voneinander vorgenommen haben, und führt sie in einem einzigen "Master"-Repository zusammen.\\

 %\begin{enumerate}
 %	\item Lese die ZIMK Seite https://www.uni-trier.de/index.php?id=68674
 %	\item git clone auf remote machine
 %\end{enumerate}
 %\textbf{\gitlab}
 %\begin{enumerate}
 %	\item mit ZIMK Kennung anmelden: \url{https://gitlab.uni-trier.de/}
 %	\item \textbf{New Project | Create blank project |}....fill in...\textbf{| Create project}
 %	\begin{center}
 %		\includegraphics[width=0.5\textwidth]{git_createProject}
 %	\end{center}
 %\end{enumerate}
 %~\\
 %\textbf{\filemanager}
 %\begin{enumerate}
 %	\item \texttt{.../Seafile/.../code/.git/config} öffnen
 %	\item folgendes eintragen:...
 %\end{enumerate}
 %alternativ über die Konsole:
 %\begin{enumerate}
 %	\item \texttt{git init}
 %	\item git remote add origin git@gitlab.uni-trier.de:vollmann/project\_iterativesolvers.git
 %	\item git add .
 %	\item git commit -m "Initial commit"
 %	\item git push -u origin master
 %\end{enumerate}
 %~\\
 %\textbf{\pycharm}\\
 %\url{https://www.jetbrains.com/help/pycharm/set-up-a-git-repository.html#e1c9b3f9}\\
 %im lokalen Ordner, auf der Homepage, merge, push etc...

\subsection{github, gitlab}
