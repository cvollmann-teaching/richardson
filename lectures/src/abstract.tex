% !TeX spellcheck = de_AT_frami
\vspace{1cm}
\renewcommand\abstractname{Übersicht}
\begin{abstract}
\noindent Leadsheet zum praktischen zweiten Teil der Veranstaltung ``Programmierung für Mathematiker''. Das Kernziel ist der Aufbau einer integrierten Arbeitsumgebung, die ein verzahntes Arbeiten mit den folgenden Komponenten ermöglicht:
\begin{itemize}
	\item Code: Entwicklung mit Python in \pycharm\ und Dokumentation mit Sphinx,
\item Text: Mathematische Textverarbeitung mit \LaTeX~in \texstudio,
\item IT/Backup: Versionskontrolle mit git~auf dem \gitlab~Server der Universität Trier und Online--Datensicherung mit \seafile\ auf Servern in RLP.
\end{itemize}
~\\
Vorbereitung auf:
\begin{itemize}
	\item Numerik-Veranstaltungen (Code)
	\item Seminararbeiten (Vortrag, Text, Code, IT/Backup)
	\item Praktikumsbericht (Text, IT/Backup)
	\item Bachelor- und Master-Arbeiten (Text, Code, IT/Backup)
	\item Doktorarbeit und Forschung (Vortrag, Text, Code, IT/Backup,...)
\end{itemize}
Projekt:
\begin{itemize}
	\item Um das Erlernte an einem Fallbeispiel praktisch anzuwenden, werden wir uns ein bestimmtes mathematisches Problem vornehmen und dieses mit geeigneten numerischen Methoden lösen.
	\item Genauer werden wir einen iterativen Löser für lineare Gleichungssysteme implementieren. Das wird bereits ein Vorgeschmack auf die spannenden Numerik-Veranstaltungen im weiteren Studiumsverlauf.
\end{itemize}
%~\\~\\
%\textbf{Quellen:} Die Inhalte sind im Wesentlichen baugleich entnommen aus:
%\begin{itemize}
%	\item \cite{Meister} für Standardresulte aus der Numerik
%	\item \cite{gumm3} und \cite{gumm2} für Grundlagen der Informatik
%	\item \cite{python3} für Python Abschnitte
%	\item \cite{latex1} für den \LaTeX--Teil
%	\item Wikipedia
%	\item Originaldokumentationen (PyCharm, sphinx, git,...)
%\end{itemize}
%~\\
%\textbf{Zugriff auf Remote Server:}
%\begin{enumerate}
%	\item VPN Verbindung mit dem Uni Netz herstellen:  \url{https://www.uni-trier.de/index.php?id=62624}
%	\item Remote Server:
%	\begin{itemize}
%		\item \textbf{Webserver Interface:} \url{http://thalassa.uni-trier.de:8080/guacamole/}\\
%		username: progprak, pwd: password\\
%		\textit{Dieser Webserver dient nur als command line interface zum eigentlichen Server syrma.uni-trier.de. Hier haben wir alle dieselben Zugangsdaten.}
%		\item	Sobald Sie sich angemeldet haben, erscheint eine Konsole mit der Aufforderung \texttt{"Login as"}. Hier brauchen Sie nun Ihren entsprechenden Benutzernamen samt Passwort gemäß der \link{https://www.math.uni-trier.de/~vollmann/prog/2021-prog-login-liste.html}{login-Liste}.
%		\item Bei der ersten Anmeldung werden Sie aufgefordert ein neues Passwort für Ihren Benutzer zu vergeben.
%	\end{itemize}
%		\item Alternativ: Sollten Sie einen \textbf{SSH Client} auf Ihrem lokalen Rechner installiert haben, so brauchen Sie den Umweg über Das Web--Interface nicht zu gehen. In diesem Fall können Sie sich sofort verbinden via: $$\texttt{ssh username@syrma.uni-trier.de}$$
%\end{enumerate}
%~\\
%\begin{center}
%	\textit{Alle organisatorischen Details entnehmen Sie bitte dem \link{https://www.math.uni-trier.de/~vollmann/prog/information.pdf}{Informationsblatt} der Veranstaltung\\ (insbesondere Abschnitt Prüfung 2).}
%\end{center}
\end{abstract}
